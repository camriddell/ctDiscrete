\newcommand{\econtexRoot}{..}
\providecommand{\econtex}{./texmf-local/tex/latex/econtex}
\providecommand{\econtexSetup}{./texmf-local/tex/latex/econtexSetup}
\providecommand{\econtexShortcuts}{./texmf-local/tex/latex/econtexShortcuts}
\providecommand{\econtexBibMake}{./texmf-local/tex/latex/econtexBibMake}
\providecommand{\econtexBibStyle}{./texmf-local/bibtex/bst/econtex}
\providecommand{\notes}{./texmf-local/tex/latex/handout}
\providecommand{\handoutSetup}{./texmf-local/tex/latex/handoutSetup}
\providecommand{\handoutShortcuts}{./texmf-local/tex/latex/handoutShortcuts}
\providecommand{\handoutBibMake}{./texmf-local/tex/latex/handoutBibMake}
\providecommand{\handoutBibStyle}{./texmf-local/bibtex/bst/handout}

  
\documentclass[titlepage,abstract]{\econtex}\newcommand{\textname}{Methods}
\usepackage{\econtexSetup}\usepackage{\econtexShortcuts}
\newcommand{\PTremark}[2][]{PT: {\it #2}}
\newcommand{\CDCremark}[2][]{CDC: {\it #2}}

\newenvironment{PTinserted}%
  {\noindent\ignorespaces\color{red}}%
  {\noindent\ignorespacesafterend}%

\newenvironment{CDCinserted}%
  {\noindent\ignorespaces\color{green}}%
  {\noindent\ignorespacesafterend}%

\newenvironment{PTmodified}%
  {\noindent\ignorespaces\color{red}}%
  {\noindent\ignorespacesafterend}%

\newenvironment{CDCmodified}%
  {\noindent\ignorespaces\color{red}}%
  {\noindent\ignorespacesafterend}%

\newcommand\PTcrossedout[1]{\color{red}{\sout{#1}}\color{black}}
\newcommand\CDCcrossedout[1]{\color{red}{\sout{#1}}\color{black}}

\let\PTdeleted\comment
\let\CDCdeleted\comment

% immature stuff
\newcommand{\done}{\ding{52}}
\newcommand{\notdone}{\ding{56}}
\newcommand{\cut}{\ding{36}}
\newcommand{\love}{\ding{170}}


% Nice tables
%\usepackage{booktabs}%defines\tabcolsep

% Modify commands defined in CDCShortcuts.sty
\renewcommand{\tSS}{}
    %\providecommand{\tSS}{\ensuremath{t}}
\renewcommand{\urate}{\ensuremath{\mu}}
    %\providecommand{\urate}{\ensuremath{\mho}}
\renewcommand{\erate}{(1-\urate)}
    %\providecommand{\erate}{\ensuremath{\cancel{\mho}}}
\renewcommand{\mLev}{\ensuremath{m}}
    %\providecommand{\mLev}{\ensuremath{\pmb{m}}}
    \renewcommand{\cLev}{\ensuremath{c}}
        %\providecommand{\cLev}{\ensuremath{\pmb{c}}}
\renewcommand{\aLev}{\ensuremath{a}}
    %\providecommand{\aLev}{\ensuremath{\pmb{a}}}
\renewcommand{\bLev}{\ensuremath{b}}
    %\providecommand{\bLev}{\ensuremath{\pmb{b}}}
\renewcommand{\mTarg}{\ensuremath{m}}
\newcommand{\nablaTarg}{\ensuremath{\nabla}}
\renewcommand{\aleph}{\ensuremath{\psi}}
\newcommand{\CEA}{\textsc{cea}}
\newcommand{\EU}{\textsc{eu}}
\newcommand{\VectorX}{\ensuremath{\pmb{X}}}
\newcommand{\VectorCoeff}{\ensuremath{\pmb{\gamma}}}

\usepackage{txfonts}


\pdfinclusioncopyfonts=1% graphics fonts included even if available on disk
                        % fonts on disk with same name might be different


\bibliographystyle{\econtexBibStyle}\begin{document}

\hfill{\tiny \textname}

\title{A Tractable Model \\ of Buffer Stock Saving: \\ Methods}

\begin{verbatimwrite}{\textname.title}
A Tractable Model of Buffer Stock Saving
\end{verbatimwrite}

\author{Christopher D. Carroll\authNum \and Patrick Toche\authNum}%

\keywords{risk, uncertainty, precautionary saving, buffer stock saving}%

\jelclass{C61, D11, E24}%

\maketitle%
 
\begin{abstract}
  This document briefly describes the methods used to solve the model described in the main paper.
\end{abstract}

\vspace{0.15in}

\begin{small}
\parbox{\textwidth}{
\begin{center}
\begin{tabbing} 
\texttt{Archive:~} \= \= \url{http://econ.jhu.edu/people/ccarroll/papers/ctDiscrete.zip} \kill \\  % This line establishes the locations of the tabs, but is not printed because of the \kill directive
\texttt{~~~~PDF:~} \> \> \url{http://econ.jhu.edu/people/ccarroll/papers/ctDiscrete.pdf} \\
\texttt{~~~~Web:~} \> \> \url{http://econ.jhu.edu/people/ccarroll/papers/ctDiscrete/}    \\ 
\texttt{Archive:~} \> \> \url{http://econ.jhu.edu/people/ccarroll/papers/ctDiscrete.zip} \\
\texttt{~~~~~~~~~} \> \> {\it (Contains Mathematica and Matlab code solving the model)}
\end{tabbing}
\end{center}
}
\end{small}

\begin{authorsinfo}
\name{Carroll: \href{mailto:ccarroll@jhu.edu}{\texttt{ccarroll@jhu.edu}}, Department of Economics, Johns Hopkins University, Baltimore Maryland 21218, USA; and National Bureau of Economic Research.  \url{http://econ.jhu.edu/people/ccarroll}
\name{Toche: \href{mailto:contact@patricktoche.com}{\texttt{contact@patricktoche.com}}, Christ Church, Oxford OX1 1DP, UK.}}
\end{authorsinfo}%

\thanks{Thanks to several generations of Johns Hopkins University graduate students who helped correct many errors and infelicities in earlier versions of the material presented here.}%

\titlepagefinish%

\section{Numerical Solution}

\subsection{The Consumption Function}

To solve the model by the method of {\it reverse shooting},
%\footnote{See \cite{judd:book} for a presentation of shooting methods of solution for numerical difference and differential equations.} 
we need $\cRatE_{t}$ as a function of $\cRatE_{t+1}$.
\begin{eqnarray*}
         \left(\frac{\cRatE_{t+1}}{\cRatE_{t}}\right) & = & \PGro^{-1} (\Rfree\Discount)^{1/\CRRA} \left\{1+\urate\left[\left(\frac{\cRatE_{t+1}}{\cU_{t+1}}\right)^{\CRRA}-1\right]\right\}^{1/\CRRA}
\\       \cRatE_{t} & = & \left(\frac{\cRatE_{t+1}}{\PGro^{-1} (\Rfree\Discount)^{1/\CRRA} \left\{1+\urate\left[\left(\frac{\cRatE_{t+1}}{\MPCU (\mRatE_{t+1}-1)}\right)^{\CRRA}-1\right]\right\}^{1/\CRRA} }  \right)
\\        & = & \PGro (\Rfree\Discount)^{-1/\CRRA} \cRatE_{t+1}\left\{1+\urate\left[\left(\frac{\cRatE_{t+1}}{\MPCU (\mRatE_{t+1}-1)}\right)^{\CRRA}-1\right]\right\}^{-1/\CRRA}        .
\end{eqnarray*}


We also need the reverse shooting equation for $\mRatE_{t}$:
\begin{eqnarray*}
%        \mRatE_{t+1} & = & \Rnorm(\mRatE_{t}-\cRatE_{t})+1  \\
        \mRatE_{t} & = & \Rnorm^{-1} (\mRatE_{t+1}-1)+\cRatE_{t} .
\end{eqnarray*}

The reverse shooting approximation will be more accurate if we use it to obtain estimates
of the marginal propensity to consume as well.  These are obtained 
by differentiating the consumption Euler equation with respect to $m_{t}$:
\begin{eqnarray}
  \uP(\cFunc^{e}(m_{t})) & = & \overbrace{\Rnorm \Discount \PGro^{1-\CRRA}}^{\beth} \Ex_{t}[\uP(\cFunc^{\bullet}(m_{t+1}))] \notag
\\  \uFunc^{\prime\prime}(\cFunc^{e}(m_{t}))\MPCFunc^{e}(m_{t}) & = & \beth  \Rnorm (1-\MPCFunc^{e}(m_{t}))\Ex_{t}[\uFunc^{\prime\prime}(\cFunc^{\bullet}(m_{t+1}))\MPCFunc^{\bullet}(m_{t+1})] \label{eq:dEuler}
\end{eqnarray}
so that defining $\MPCE_{t} = \MPCFunc^{e}(m_{t})$ we have 
\begin{eqnarray}
 \MPCE_{t} & = &  (1-\MPCE_{t}) \underbrace{\beth \Rnorm  (1/\uPP(\cRatE_{t}))\Ex_{t}\left[\uPP(c^{\bullet}_{t+1})\MPC^{\bullet}_{t+1}\right]}_{\equiv \natural_{t+1}}  \label{eq:natural}
\\ (1+\natural_{t+1})\MPCE_{t} & = & \natural_{t+1} \label{eq:naturalMPC} 
\\ \MPCE_{t} & = & \left(\frac{\natural_{t+1}}{1+\natural_{t+1}}\right) \label{eq:naturalSolved}
.
\end{eqnarray}
\newcommand{\Mma}{{\it Mathematica}~}

At the target level of $\mRatE$,
\begin{eqnarray*}
  \overbrace{(1/\uPP(\cTarg^{e}))\Ex_{t}\left[\uPP(c^{\bullet})\MPC^{\bullet}\right]}^{\check{\natural} / \Rnorm \beth} & = & \erate \overbrace{(\uPP(\cTarg^{e})/\uPP(\cTarg^{e}))}^{=1}\MPCE+\urate (\uPP(\cTarg^{u})/\uPP(\cTarg^{e}))\MPCU
\end{eqnarray*}
so that 
\begin{eqnarray}
  \check{\natural} & = &  \beth \Rnorm (\erate \MPCE + \urate (\cTarg^{u}/\cTarg^{e})^{-\CRRA-1} \MPCU)
\end{eqnarray}
yielding from \eqref{eq:naturalMPC} a quadratic equation in $\MPCE$:
\begin{eqnarray}
  \left(1+\beth \Rnorm (\erate \MPCE + \urate (\cTarg^{u}/\cTarg^{e})^{-\CRRA-1}  \MPCU) \right)\MPCE & = & \beth \Rnorm (\erate \MPCE + \urate (\cTarg^{u}/\cTarg^{e})^{-\CRRA-1} \MPCU)
\end{eqnarray} 
which has one solution for $\MPCE$ in the interval $[0,1]$, which is the MPC at target wealth.\footnote{The 
\Mma code constructs this derivative and solves the quadratic equation analytically; the Matlab code simply copies
the analytical formula generated by \Mma.}

The limiting MPC as consumption approaches zero, $\bar{\MPC}^{e},$ will also be useful; this is obtained
by noting that utility in the employed state next year becomes asymptotically irrelevant as $\cRatE_{t}$ approaches zero, so that 
\begin{eqnarray*}
  \lim_{\cRatE_{t} \rightarrow 0}  \overbrace{ \beth \Rnorm \MPCE_{t+1} \left(\erate (\cRatE_{t+1}/\cRatE_{t})^{-\CRRA-1}  + \urate (\cU_{t+1}/\cRatE_{t})^{-\CRRA-1}\MPCU\right)}^{\natural_{t+1}} & = & \beth \Rnorm \urate (\cU_{t+1}/\cRatE_{t})^{-\CRRA-1}\MPCU
\\  & = & \beth \Rnorm \urate (\MPCU \Rnorm \aE_{t}/(\aE_{t}(\bar{\MPC}^{e}/(1-\bar{\MPC}^{e})))^{-\CRRA-1})\MPCU
\\ & = & \beth \Rnorm \urate (\MPCU \Rnorm ((1-\bar{\MPC}^{e})/\bar{\MPC}^{e}))^{-\CRRA-1}\MPCU
%\\ & = & \beth \Rnorm \urate (\MPCU \Rnorm)^{-\CRRA-1} \MPCU
\end{eqnarray*}
so that from \eqref{eq:naturalSolved} we have 
\begin{eqnarray}
  \bar{\MPC}^{e} \equiv \lim_{\mRat_{t} \rightarrow 0} \MPCFunc^{e}(m_{t}) & = & \left(\frac{
\beth \Rnorm \urate (\MPCU \Rnorm ((1-\bar{\MPC}^{e})/\bar{\MPC}^{e}))^{-\CRRA-1}\MPCU
}{1+
\beth \Rnorm \urate (\MPCU \Rnorm ((1-\bar{\MPC}^{e})/\bar{\MPC}^{e}))^{-\CRRA-1}\MPCU
}\right)
\end{eqnarray}
which implicitly defines $\bar{\MPC}^{e}$.  After parameter values
have been defined a numerical rootfinder can calculate a solution
almost instantly.

Finally, it will be useful to have an estimate of the curvature
(second derivative) of the consumption function at the target.  This
can be obtained by a procedure analogous to the one used to obtain the
MPC: differentiate the differentiated Euler equation \eqref{eq:dEuler}
again and substitute the target values.  Noting that
$\MPC^{u\prime}=0$, we can obtain:
\begin{eqnarray}
(\MPCFunc^{e}_{t})^{2} \uPPP(\cFunc^{e}_{t})
+\MPCFunc_{t}^{e\prime}\uPP(\cFunc^{e}_{t})= &  & 
\beth \Rnorm \left\{(-\MPCFunc_{t}^{e\prime}) \Ex_{t}[\uPP(\cFunc^{\bullet}_{t+1})\MPCFunc^{\bullet}_{t+1}] \right. \\
& + & %\left. 
\Rnorm (1-\MPCFunc^{e}_{t})^{2}\left(\Ex_{t}[(\MPCFunc_{t+1}^{\bullet})^{2}\uPPP(\cFunc^{\bullet}_{t+1})]+\erate  \uPP(\cFunc^{e}_{t+1})\MPCFunc_{t+1}^{e\prime}\right) \nonumber
\label{eq:kappaPExpr}
\end{eqnarray}
\begin{comment} % The following attempts to derive an expression that will be useful in determining a functional form for \MPC^{e \prime} as c goes to infinity
the limit of which as $\cRatE_{t} \rightarrow \infty$ is 
\begin{eqnarray}
\lefteqn{  (\MPCE)^{2} \uPPP(\cRatE_{t}) +\MPC^{e\prime}\uPP(\cRatE_{t}) = } &  & \nonumber \\ 
& & \beth \Rnorm (-\MPC^{e\prime}) \Ex_{t}[\uPP(\cFunc^{\bullet}_{t+1})\MPC^{\bullet}_{t+1}]
 + \beth \Rnorm^{2} (1-\MPCE)^{2} (\Ex_{t}[(\MPC_{t+1}^{\bullet})^{2}\uPPP(\cFunc^{\bullet}_{t+1})]+\erate \uPP(\cRatE_{t+1})\MPC^{e\prime}) \nonumber
\end{eqnarray}
but since as $\cRatE_{t} \rightarrow \infty$ the ratio of $\cRatE/\cTarg^{u}$ approaches 1 this approaches
\begin{eqnarray}
\lefteqn{  (\MPCE)^{2} \uPPP({c}_{t}) +\MPC^{e\prime}_{t}\uPP(c_{t}) = } &  & \nonumber \\ 
& & \beth \Rnorm (-\MPC^{e\prime}_{t}) \uPP(c_{t+1})\MPC
 + \beth \Rnorm^{2} (1-\MPCE)^{2} (\MPCE)^{2}\uPPP(c_{t+1})+\erate \uPP({c}_{t+1})\MPC^{e\prime}_{t+1}) \nonumber
\end{eqnarray}
but since $c_{t+1}$ approaches $\mu c_{t}$ this can be rewritten
\begin{eqnarray}
\lefteqn{  (\MPCE)^{2} \uPPP({c}_{t}) +\MPC^{e\prime}_{t}\uPP(c_{t}) = } &  & \nonumber \\ 
& & \beth \Rnorm (-\MPC^{e\prime}_{t}) \uPP(c_{t}) \mu^{-\CRRA-1}\MPC
 + \beth \Rnorm^{2} (1-\MPCE)^{2} (\MPCE)^{2}\uPPP(c_{t})\mu^{-\CRRA-2}+\erate \uPP({c}_{t})\mu^{-\CRRA-2}\MPC^{e\prime}_{t+1}) \nonumber
\end{eqnarray}
and dividing both sides by $\uPP(c_{t})$ and realizing that $\uPPP(c)/\uPP(c) = (\CRRA+1)/c$ we have
\begin{eqnarray}
\lefteqn{  (\MPCE)^{2} (\CRRA+1)/c_{t} +\MPC^{e\prime}_{t}= } &  & \nonumber \\ 
& & \beth \Rnorm (-\MPC^{e\prime}_{t})  \mu^{-\CRRA-1}\MPC
 + \beth \Rnorm^{2} (1-\MPCE)^{2} (\MPCE)^{2}\mu^{-\CRRA-2}(\CRRA+1)/c_{t}+\erate \mu^{-\CRRA-2}\MPC^{e\prime}_{t+1}) \nonumber
\end{eqnarray}
\end{comment}
so that
\begin{eqnarray}
\MPCFunc^{e\prime}_{t}   & = & \left(\frac{\beth \Rnorm^{2} (1-\MPCFunc^{e}_{t})^{2} \left( \Ex_{t}[(\MPCFunc_{t+1}^{\bullet})^{2}\uPPP(\cFunc^{\bullet}_{t+1})] +\erate \uPP(\cFunc^{e}_{t+1}) \MPCFunc^{e \prime}_{t+1}\right)-(\MPCFunc^{e}_{t})^{2} \uPPP(\cFunc^{e}_{t})}{\uPP(\cFunc^{e}_{t})+\beth \Rnorm \Ex_{t}[\uPP(\cFunc^{\bullet}_{t+1})\MPCFunc^{\bullet}_{t+1}]  }\right) \nonumber
\end{eqnarray}
%leading to a dynamic equation for $\MPC_{t}^{e\prime}$
%\begin{eqnarray}
%\MPC_{t}^{e\prime} & = & \left(\frac{\beth \Rnorm^{2} (1-\MPCE_{t})^{2} (\Ex_{t}[(\MPC_{t+1}^{\bullet})^{2}\uPPP(c^{\bullet}_{t+1})]+\erate \uPP(\cRatE_{t+1}) -(\MPCE_{t})^{2} \uPPP(\cRatE_{t}))}{\uPP(\cRatE_{t})+\beth \Rnorm \Ex_{t}[\uPP(c^{\bullet}_{t+1})\MPC^{\bullet}_{t+1}]}\right)  \nonumber
%\end{eqnarray}
%whose limit as $\cRatE \rightarrow \infty$ is 
which can be further simplified at the target because $\MPCFunc^{e \prime}_{t}(\mTarg^{\null}) = \MPCFunc^{e \prime}_{t+1}(\mTarg^{\null}) = \MPC^{e \prime}$ so that
\begin{eqnarray}
\MPC^{e\prime} & = & \left(\frac{\beth \Rnorm^{2} (1-\MPCE)^{2}\Ex_{t}[(\MPC^{\bullet})^{2} \uPPP(c^{\bullet})] -(\MPCE)^{2} \uPPP(\cTarg^{e})}{\uPP(\cTarg^{e}) + \beth \Rnorm \Ex_{t}[\uPP(c^{\bullet})\MPC^{\bullet}]- \beth \Rnorm^{2} (1-\MPCE)^{2}\erate \uPP(\cTarg^{e})}\right) \label{eq:MPCPrimeSS}
.
\end{eqnarray}

Another differentiation of \eqref{eq:kappaPExpr} similarly allows the construction of a formula for the value of $\MPC^{e \prime\prime}$ at the target $\mTarg^{\null}$; in principle, any number of derivatives can be 
constructed in this manner.\footnote{\Mma permits the convenient computation of the analytical derivatives, and then the substitution of constant target values to obtain analytical expressions like \eqref{eq:MPCPrimeSS}.  These solutions are simply imported by hand into the Matlab code.}


\begin{comment} % There's something wrong with the argument below, but I'm not sure what.
% Anyway, this approach is superceded by the one below.
Since $-\uPPP/\uPP = (\CRRA+1) / c$, intuition suggests (and deeper analysis confirms) that $\lim_{\mRatE_{t} \rightarrow \infty} \MPCFunc^{e \prime}(\mRatE_{t})$ is well approximated by $\gamma/\mRatE_{t}$ for some constant $\gamma$.  Indeed,
since $\lim_{\mRatE_{t} \rightarrow \infty} \MPCFunc(\mRatE_{t}) = \MPC$ and the limiting difference between
$\cFunc^{e}$ and $\cFunc^{u}$ is finite, analysis like that which led to \eqref{eq:MPCPrimeSS} should convince
the reader that
\begin{eqnarray}
\lim_{\mRatE_{t} \rightarrow \infty} \MPCFunc^{e\prime}(\mRatE_{t}) & = & \left(\frac{\uPPP(\cRatE_{t}) (\beth \Rnorm^{2} (1-\MPC)^{2}\MPC^{2}  -\MPC^{2}) }{\uPP(\cTarg^{e})(1-\beth \Rnorm^{2} (1-\MPC)^{2}\erate  + \beth \Rnorm \MPC}\right)
\\ & = & -(1/\cRatE_{t})\left(\frac{(\CRRA+1) (\beth \Rnorm^{2} (1-\MPC)^{2}-1)\MPC^{2}  }{1-\beth \Rnorm^{2} (1-\MPC)^{2}\erate  + \beth \Rnorm \MPC}\right)
\\ & = & -(1/\mRatE_{t})\left(\frac{(\CRRA+1) (\beth \Rnorm^{2} (1-\MPC)^{2}-1)\MPC  }{1-\beth \Rnorm^{2} (1-\MPC)^{2}\erate  + \beth \Rnorm \MPC}\right)
.
\end{eqnarray}
\end{comment}


Reverse shooting requires us to solve separately for an approximation to the consumption function above the steady state and 
another approximation below the steady state.  Using the approximate steady-state $\MPCE$ and $\MPC^{e\prime}$ 
obtained above, we begin by picking a very small number for $\blacktriangle$ and then creating a Taylor
approximation to the consumption function near the steady state:
\begin{eqnarray}
  \mRatE_{\Alt{t}} & = & \mTarg^{\null} + \blacktriangle  \label{eq:revshootmstart} 
\\ \tilde{\mathbf{c}}(\blacktriangle) & = & \cTarg^{e} + \blacktriangle \MPCE + (\blacktriangle^{2}/2) \MPC^{e\prime}+ (\blacktriangle^{3}/6) \MPC^{e\prime\prime} \label{eq:revshootcstart}
%\\ \MPCE_{\Alt{t}} & = & \hat{\MPCFunc}^{e}(\blacktriangle)
\end{eqnarray}
and then iterate the reverse-shooting equations until we reach some period $n$ in
which $\mRatE_{\Alt{t}-n}$ escapes some pre-specified interval $[\ushort{\mRat}^{e},\bar{\mRat}^{e}]$ (where the natural 
value for $\ushort{\mRat}^{e}$ is 1 because this is the $\mRat$ that would be owned by a consumer who had saved
nothing in the prior period and therefore is below any feasible value of $\mRat$ that could be realized by an 
optimizing consumer).  This generates a sequence of points all of which
are on the consumption function.  A parallel procedure (substituting $-$ for
$+$ in \eqref{eq:revshootmstart} and where appropriate in \eqref{eq:revshootcstart}) generates the sequence of points for the approximation below the 
steady state.  Taken together with the already-derived characterization of the function
at the target level of wealth, these points constitute the basis for an interpolating approximation to the
consumption function on the interval $[\ushort{\mRat}^{e},\bar{\mRat}^{e}]$.



\subsection{The Value Function}

As a preliminary, note that since $\uFunc(xy)=\uFunc(x)y^{1-\CRRA}$, value for an unemployed consumer is 
\begin{eqnarray}
  \VFunc_{t}^{u} & = & \uFunc(C_{t}^{u})+\Discount \uFunc(C_{t+1}^{u}) + \Discount^{2} \uFunc(C_{t+2}^{u})+...
\\ & = & \uFunc(C_{t}^{u})\left(1+\Discount \{(\Rfree \Discount)^{1/\CRRA}\}^{1-\CRRA} + \Discount^{2}\left\{(\Rfree \Discount)^{2/\CRRA}\right\}^{1-\CRRA}+...\right)
\\ & = & \uFunc(C_{t}^{u})\underbrace{\left(\frac{1}{1-\Discount (\Rfree \Discount)^{(1/\CRRA)-1}}\right)}_{\equiv \mathfrak{v}}
\end{eqnarray}
where the RIC guarantees that the denominator in the fraction is a positive number.  

From this we can see that value for the normalized problem is similarly:
\begin{eqnarray}
  \vFunc^{u}(m_{t}) & = & \uFunc(\MPCU m_{t}) \mathfrak{v}
.
\end{eqnarray}

Turning to the problem of the employed consumer, we have
\begin{eqnarray}
  \vFunc^{e}(m_{t}) & = & \uFunc(\cRatE_{t})+\Discount \PGro^{1-\CRRA} \Ex_{t}[\vFunc^{\bullet}(m_{t+1})]
\end{eqnarray}
and at the target level of market resources this will be unchanging for a consumer who
remains employed so that 
\begin{eqnarray}
  \vTarg^{e} & = & \uFunc(\cTarg^{e})+\Discount \PGro^{1-\CRRA} \left(\erate \vTarg^{e} + \urate \vFunc^{u}(\aE \Rnorm)\right)
\\ (1-\Discount \PGro^{1-\CRRA} \erate) \vTarg^{e} & = & \uFunc(\cTarg^{e})+\Discount \PGro^{1-\CRRA} \urate \vFunc^{u}(\aE \Rnorm)
\\ \vTarg^{e} & = & \left(\frac{\uFunc(\cTarg^{e})+\Discount \PGro^{1-\CRRA} \urate \vFunc^{u}(\aE \Rnorm)}{(1-\Discount \PGro^{1-\CRRA} \erate) }\right)
.
\end{eqnarray}

Given these facts, our recursion for generating a sequence of points on the consumption
function can be used at the same time to generate corresponding points on the value function from
\begin{eqnarray}
  \vE_{t} & = & \uFunc(\cRatE_{t})+\Discount \PGro^{1-\CRRA} \left(\erate \vE_{t+1} + \urate \vFunc^{u}(\aE_{t} \Rnorm)\right)
\end{eqnarray}
with the first iteration point generated by numerical integration from 
\begin{eqnarray}
  v^{e}_{\Alt{t}} & = & \vTarg^{e}+\int_{0}^{\blacktriangle} \uP(\tilde{\mathbf{c}}(\bullet)) d\bullet
\end{eqnarray}



\section{The Algorithm}

With the above results in hand, the model is solved and the various
functions constructed as follows.  Define $\star_{t} =
\{\mRatE_{t},\cRatE_{t},\MPCE_{t},\vE_{t},\MPC_{t}^{e\prime}\}$ as a vector of points that
characterizes a particular situation that an optimizing employed
household might be in at any given point in time.  Using the backwards-shooting 
functions derived above, for any point $\star_{\Alt{t}}$ we can construct the
sequence of points that must have led up to it: $\star_{\Alt{t}-1}$ and
$\star_{\Alt{t}-2}$ and so on.  And using the approximations near the
steady state like \eqref{eq:revshootcstart}, we can construct 
a vector-valued function $\pmb{\circ}(\blacktriangle)$ that generates, 
e.g., $\{\mTarg^{\null}+\blacktriangle,\tilde{\mathbf{c}}(\blacktriangle), ... \}$.


Now define an operator $\cdots$ as follows: $\cdots$ applied to some
starting point $\star_{t}$ uses the backwards dynamic equations
defined above to produce a vector of points
$\star_{t-1},\star_{t-2},...$ consistent with the model until the
$\mRatE_{t-n}$ that is produced goes outside of the pre-defined bounds 
$[\ushort{\mRat}^{e},\bar{\mRat}^{e}]$ for solving the problem.

We can merge the points below the steady state with the steady state
with the points above the steady state to produce $\dddot{\star} =
\cdots(\pmb{\circ}(-\varepsilon)) \cup \pmb{\circ}(0) \cup
\cdots(\pmb{\circ}(\varepsilon)) $.  These points can then be used to
generate appropriate interpolating approximations to the consumption
function and other desired functions.

Designate, e.g., the vector of points on the consumption function
generated in this manner by $\dddot{\star}[c]$, so that 
\begin{eqnarray}
   \{\dddot{\star}[m],\{\dddot{\star}[c],\dddot{\star}[\MPC^{e}],\dddot{\star}[\MPC^{e\prime}]\}^{\intercal}\}^{\intercal} & = & 
\begin{pmatrix}
m[1] & \{c[1],\MPC^{e}[1],\MPC^{e\prime}[1]\} \\
m[2] & \{c[2],\MPC^{e}[2],\MPC^{e\prime}[2]\} \\
...  & ...              \\
m[N] & \{c[N],\MPC^{e}[N],\MPC^{e\prime}[N]\} \\
\end{pmatrix} \label{eq:cFuncMat}
\end{eqnarray}
where $N$ is the number of points that have been generated by the merger of the backward
shooting points described above.  

The object \eqref{eq:cFuncMat} is not an arbitrary example; it reflects a set of values that 
uniquely define a fourth order piecewise polynomial spline such that at every point in the 
set the polynomial matches the level and first derivative included in the list.  Standard
numerical mathematics software can produce the interpolating function with this input; 
for example, the syntax in \Mma is simply
\begin{eqnarray}
  \mathtt{cE} & = & \mathtt{Interpolation}[\{\dddot{\star}[m],\{\dddot{\star}[c],\dddot{\star}[\MPC^{e}],\dddot{\star}[\MPC^{e\prime}]\}^{\intercal}\}^{\intercal}].
\end{eqnarray}
which creates a function $\texttt{cE}$ that is a $\mathbf{C}^4$ interpolating polynomial
connecting these points.




The reverse shooting algorithm terminates at some finite maximum point $\bar\mRat$, but for completeness 
it is useful to have an approximation to the consumption function that is reasonably well behaved
for any $\mTarg^{\null}$ no matter how large.\footnote{An extrapolation of the approximating interpolation will 
not perform well; a polynomial approximation will inevitably ``blow up'' if evaluated at large enough $\mTarg^{\null}$.}

Since we know that the consumption function in the presence of uncertainty asymptotes to the 
perfect foresight function, we adopt the following approach.  Defining the level of precautionary
saving as\footnote{Mnemonic: This is the amount of consumption that is canceled as a result of uncertainty.}
\newcommand{\psav}{\cancel{c}}
\newcommand{\psavFunc}{\ensuremath{\cancel{\cFunc}}}
\begin{eqnarray}
  \label{eq:pSavFunc}
  \psavFunc(\mRat) & = & \bar{\cFunc}(\mRat)-\cFunc(\mRat),
\end{eqnarray}
we know (see the discussion below in appendix section \ref{sec:PGroGEQRfree}) that 
\begin{eqnarray}
  \lim_{\mRat \rightarrow \infty} \psavFunc(\mRat) = 0
.
\end{eqnarray}

Defining $\vec{\mRat}=m-\bar{\mRat}$, a convenient functional form to postulate for the
propensity to precautionary-save is
\begin{eqnarray}
  \psavFunc(\mRat) & = & e^{\phi_{0}-\phi_{1} \vec{\mRat}}+e^{\gamma_{0}-\gamma_{1} \vec{\mRat}}
\end{eqnarray}
with derivatives
\begin{eqnarray}
    \psavFunc^{\prime}(\mRat) & = & -\phi_{1} e^{\phi_{0}-\phi_{1} \vec{\mRat}} - \gamma_{1} e^{\gamma_{0}-\gamma_{1} \vec{\mRat}}
\\  \psavFunc^{\prime\prime}(\mRat) & = & \phantom{-}\phi_{1}^{2} e^{\phi_{0}-\phi_{1} \vec{\mRat}} + \gamma_{1}^{2} e^{\gamma_{0}-\gamma_{1} \vec{\mRat}}
\\  \psavFunc^{\prime\prime\prime}(\mRat) & = & -\phi_{1}^{3} e^{\phi_{0}-\phi_{1} \vec{\mRat}} - \gamma_{1}^{3} e^{\gamma_{0}-\gamma_{1} \vec{\mRat}}
.
\end{eqnarray}

Evaluated at $\bar{\mRat}$ (for which $\psavFunc$ and its derivatives will have numerical values
assigned by the reverse-shooting solution method described above), this is a system of four equations in four unknowns and, though nonlinear, can be easily solved for 
values of the $\phi$ and $\gamma$ coefficients that match the level and first three derivatives
of the ``true'' $\psavFunc$ function.\footnote{The exact symmetry in
  the treatment of $\gamma$ and $\phi$ means that there will actually
  be two symmetrical solutions; either can be used.}




\section{Modified Formulas For Case Where $\PGro \geq \Rfree$} \label{sec:PGroGEQRfree}

The text asserts that if $\PGro < \Rfree$ the consumption function for a finite-horizon employed consumer approaches
the $\bar{\cFunc}_{t}(\mRat)$ function that is optimal for a perfect-foresight
consumer with the same horizon,
\begin{eqnarray}
  \lim_{\mRat \uparrow \infty} \bar{\cFunc}_{t}(\mRat) - \cFunc_{t}(\mRat) & = & 0
.
\end{eqnarray}

This proposition can be proven by careful analysis of the consumption Euler equation,
noting that as $\mRat$ approaches infinity the proportion of consumption will
be financed out of (uncertain) labor income approaches zero, and that the magnitude
of the precautionary effect is proportional to the square of the proportion of such
consumption financed out of uncertain labor income.

A footnote also claims that for employed consumers, $\cFunc(\mRat)$
approaches a different, but still well-defined, limit even if $\PGro
\geq \Rfree$, so long as the impatience condition holds. It turns out that the limit 
in question is the one defined by the solution to a perfect foresight problem with 
liquidity constraints.  A semi-analytical solution does exist in this case, but it 
is omitted.  %A continuous-time treatement can be found in \cite{parkLiqConstrContinuous}.

\end{document}
